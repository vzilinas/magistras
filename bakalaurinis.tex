\documentclass{VUMIFPSbakalaurinis}
\usepackage{algorithmicx}
\usepackage{algorithm}
\usepackage{algpseudocode}
\usepackage{amsfonts}
\usepackage{amsmath}
\usepackage{bm}
\usepackage{caption}
\usepackage{color}
\usepackage{float}
\usepackage{graphicx}
\usepackage{listings}
\usepackage{subfig}
\usepackage{wrapfig}

\usepackage{enumitem}
\setitemize{noitemsep,topsep=0pt,parsep=0pt,partopsep=0pt}
\setenumerate{noitemsep,topsep=0pt,parsep=0pt,partopsep=0pt}

\hbadness=5000
% Titulinio aprašas
\university{Vilniaus universitetas}
\faculty{Informatikos institutas}
\department{Programų sistemų katedra}
\papertype{Magistro baigiamajo darbo planas}
\title{Srautinio apdorojimo sistemų balansavimas taikant mašininio mokymosi algoritmus}
\titleineng{Balancing stream processing systems using machine learning algorithms}
\author{Vytautas Žilinas}
\supervisor{Partn. Doc. Andrius Adamonis}
\reviewer{Prof. Dr. Aistis Raudys}
\date{Vilnius – \the\year}

% Nustatymai
% \setmainfont{Palemonas}   % Pakeisti teksto šriftą į Palemonas (turi būti įdiegtas sistemoje)
\bibliography{bibliografija}

\begin{document} 
\maketitle

\cleardoublepage\pagenumbering{arabic}
\setcounter{page}{2}

\section{Tyrimo aktualumas ir naujumas}

Šiame magistriniame darbe bus nagrinėjami neuroniniai tinklai, jų algoritmai ir  pritaikymas srautinio apdorojimo sistemų balansavimui. 

JUODRAŠTIS:

Big data ir realiu laiku apdorojimo duomenų svarba.
Srautinio apdorojimo sistemų naudojimas realiame pasaulyje.
Konfigūruojamų elementų kiekis srautinio apdorojimo sistemose.
Konfigūravimo svarba neapibrėžtuose apkrovose.
Mašininio mokymosi algoritmų privalumai palyginus su žmonėmis.
Mašininio mokymosi algoritmų pritaikymas balansavime naujumas.


\subsection{Darbo tikslas}
Ištirti mašininio mokymosi tinkamumą srautinio apdorojimo sistemų balansavimui. 

\subsection{Uždaviniai}
\begin{enumerate}
    \item Pasirinkti srautinio apdorojimo sistemos metrikas, kurios bus naudojamos balansavimui.
    \item Išanalizuoti "Reinforcement Learning" algoritmus ir pasirinkti tinkamą tyrimui.
    \item Pritaikyti pasirinktą mašininio mokymosi algoritmą srautinio apdorojimo sistemos balansavimui.
    \item Atlikti eksperimentą ir palyginti rezultatą su alternatyvomis - standartinė konfigūracija, reaktyvus balansavimas. 
\end{enumerate}

\subsection{Laukiami rezultatai}
\begin{enumerate}
    \item Pritaikant skirtingas apkrovas srautinio apdorojimo sistemoms, surinktos ir surikiuotos metrikos, reikalingos "Heron" srautinio apdorojimo sistemos balansavimui. 
    \item Pasirinktas "Reinforcement Learning" mašininio mokymosi algoritmas.
    \item Pasirinktas mašininio mokymosi algoritmas pritaikytas "Heron" srautinio apdorojimo sistemų balansavimui.  
    \item Atliktas eksperimentas naudojant sukurtą balansavimo implementaciją ir srautinio apdorojimo sistemos dirbtines apkrovas.
    \item Eksperimento rezultatai palyginti su standartine konfigūraciją ir reaktyviu balansavimu.
\end{enumerate}
 
\printbibliography[heading=bibintoc] 

\end{document}
